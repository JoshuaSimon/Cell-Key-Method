%%%%%%%%%%%%%%%%%%%%%%%%%%%%%%%%%%%%%%%%%%%%%%%%%%%%%%%%%%%%%%%%%%%%%%%%%%%%%%%%%%%%%%%%%%%%%%%%%%%%%%%%%%%%%%%%%%%%%%%%%%%%%%%%
%%%%%%%%%%%%%%%%%%%%%%%%%%%%%%%%%%%%%%%% Vorlage für Statistik-Hausarbeiten: Kopfdatei %%%%%%%%%%%%%%%%%%%%%%%%%%%%%%%%%%%%%%%%%
%%%%%%%%%%%%%%%%%%%%%%%%%%%%%%%%%%%%%%%%%%%%%%%%%%%%%%%%%%%%%%%%%%%%%%%%%%%%%%%%%%%%%%%%%%%%%%%%%%%%%%%%%%%%%%%%%%%%%%%%%%%%%%%%

%%%%%%%%%%%%%%%%%%%%%%%%%%%%%%%%%%%%%%%%%%%%%%%%%%%%%%%%%%%%%%%%%%%%%%%%%%%%%%%%%%%%%%%%%%%%%%%%%%%%%%%%%%%%%%%%%%%%%%%%%%%%%%%%
%%%%%%%%%%%%%%%%%%%%%%%%%%%%%%%%%%%%%%%%% Vorlage für Statistik-Hausarbeiten: Header %%%%%%%%%%%%%%%%%%%%%%%%%%%%%%%%%%%%%%%%%%%
%%%%%%%%%%%%%%%%%%%%%%%%%%%%%%%%%%%%%%%%%%%%%%%%%%%%%%%%%%%%%%%%%%%%%%%%%%%%%%%%%%%%%%%%%%%%%%%%%%%%%%%%%%%%%%%%%%%%%%%%%%%%%%%%


%%% Basics

\documentclass[bibtotoc, 12pt, numbers=endperiod, openbib]{scrartcl}%
\usepackage[a4paper, left=30mm, right=30mm, top=20mm, bottom=20mm, includefoot]{geometry}% Seitenränder
\usepackage[utf8]{inputenc}% in TeXnicCenter bitte umändern in: \usepackage[latin1]{inputenc}
\usepackage[T1]{fontenc}%
\usepackage[ngerman]{babel}% in englischsprachigen Arbeiten: \usepackage[english]{babel}
\usepackage{mathptmx}%
\usepackage{amsmath, amsfonts, amssymb}%
\usepackage{graphicx}%


%%%%%%%%%%%%%%%%%%%%%%%%%%%%%%%%%%%%%%%%%%%%%%%%%%%%%%%%%%%%%%%%%%%%%%%%%%%%%%%%%%%%%%%%%%%%%%%%%%%%%%%%%%%%%%%%%%%%%%%%%%%%%%%%

%%% Aussehen allgemein

\usepackage{lmodern}% Schriftart lmodern
\usepackage{setspace}% Zeilenabstand
\pagenumbering{arabic}% arabische Seitenzahlen
\setkomafont{sectioning}{\bfseries}% Serifenschrift für Überschriften
%\usepackage[small]{titlesec}%% verkleinert die Überschriften etwas
%\usepackage[singlelinecheck=0]{caption}% linksbündige Abbildungsüberschriften


%%%%%%%%%%%%%%%%%%%%%%%%%%%%%%%%%%%%%%%%%%%%%%%%%%%%%%%%%%%%%%%%%%%%%%%%%%%%%%%%%%%%%%%%%%%%%%%%%%%%%%%%%%%%%%%%%%%%%%%%%%%%%%%%

%%% Zitate und Literaturverzeichnis

% Zitate mittels \cite[Seitenzahl]{kuerzel}
\usepackage[noadjust]{cite}%
\renewcommand\citeleft{}% Zeichen links vom Zitat, z.B. (, ggf. in die Klammer einfügen
\renewcommand\citeright{}% Zeichen rechts vom Zitat, z.B. ), ggf. in die Klammer einfügen
\renewcommand\citemid{:  }% nach der Seitenzahl folgt ": ", nach Belieben veränderbar

% Literaturverzeichnis
\makeatletter%
\renewcommand{\@biblabel}[1]{}% Literaturverzeichnis nicht nummeriert, keine Klammern um nicht vorhandene Zahlen
\makeatother%

% URLs
\usepackage[hyphens]{url}% URL im Literaturverzeichnis
\urlstyle{same}% Schriftart der URL


%%%%%%%%%%%%%%%%%%%%%%%%%%%%%%%%%%%%%%%%%%%%%%%%%%%%%%%%%%%%%%%%%%%%%%%%%%%%%%%%%%%%%%%%%%%%%%%%%%%%%%%%%%%%%%%%%%%%%%%%%%%%%%%%

%%% Häufig benötigte und nützliche Pakete

% Tabellen
\usepackage{booktabs}% Linien in Tabellen: \toprule (oben), \midrule (innerhalb der Tabelle), \bottomrule (unten)
\usepackage{longtable}% für Tabellen mit Seitenumbruch
\usepackage{ltxtable}% longtable und tabularx (Tabellen über mehrere Seiten, Gesamtbreite einstellbar)
\usepackage{dcolumn}% Tabellenspalte am Dezimaltrenner ausrichten (statt l/c/r): Format D{Dezimaltrenner in Tabelle}{Dezimaltrenner, der ausgegeben werden soll}{Dezimalstellen}

\usepackage{pdflscape}% Querformat ermöglichen
\usepackage{ziffer}% kein Abstand nach Dezimaltrenner

%%%%%%%%%%%%%%%%%%%%%%%%%%%%%%%%%%%%%%%%%%%%%%%%%%%%%%%%%%%%%%%%%%%%%%%%%%%%%%%%%%%%%%%%%%%%%%%%%%%%%%%%%%%%%%%%%%%%%%%%%%%%%%%%

\usepackage[section]{placeins} %prevent floats from being moved over it
\usepackage{flafter} %used to force floats to appear after they are defined
\usepackage{float}

\usepackage{setspace}

% Packages für Code.
\usepackage{listings}
\usepackage{xcolor}

\definecolor{codegreen}{rgb}{0,0.6,0}
\definecolor{codegray}{rgb}{0.5,0.5,0.5}
\definecolor{codepurple}{rgb}{0.58,0,0.82}
\definecolor{backcolour}{rgb}{0.95,0.95,0.92}

\lstdefinestyle{mystyle}{
    backgroundcolor=\color{backcolour},   
    commentstyle=\color{codegreen},
    keywordstyle=\color{magenta},
    %numberstyle=\tiny\color{codegray},
    stringstyle=\color{codepurple},
    basicstyle=\ttfamily\footnotesize,
    breakatwhitespace=false,         
    breaklines=true,                 
    captionpos=b,                    
    keepspaces=true,                 
    numbers=left,                    
    numbersep=5pt,                  
    showspaces=false,                
    showstringspaces=false,
    showtabs=false,                  
    tabsize=2
}

\lstset{style=mystyle}


% Theorems.
\newtheorem{example}{Beispiel}
\newtheorem{definition}{Definition}% Header, kann unverändert übernommen werden
\hyphenation{}% hier nicht korrekte Wörter eingeben: Bindestrich an Trennstellen, Wörter durch Komma getrennt (keine Umlaute)

\begin{document}%
%%%%%%%%%%%%%%%%%%%%%%%%%%%%%%%%%%%%%%%%%%%%%%%%%%%%%%%%%%%%%%%%%%%%%%%%%%%%%%%%%%%%%%%%%%%%%%%%%%%%%%%%%%%%%%%%%%%%%%%%%%%%%%%%
%%%%%%%%%%%%%%%%%%%%%%%%%%%%%%%%%%%%%%%%%%%%%%%%%%%%%%%%%%%%%%%%%%%%%%%%%%%%%%%%%%%%%%%%%%%%%%%%%%%%%%%%%%%%%%%%%%%%%%%%%%%%%%%%

%%% Titelseite
%%%%%%%%%%%%%%%%%%%%%%%%%%%%%%%%%%%%%%%%%%%%%%%%%%%%%%%%%%%%%%%%%%%%%%%%%%%%%%%%%%%%%%%%%%%%%%%%%%%%%%%%%%%%%%%%%%%%%%%%%%%%%%%%
%%%%%%%%%%%%%%%%%%%%%%%%%%%%%%%%%%%%%%%% Vorlage für Statistik-Hausarbeiten: Titelseite %%%%%%%%%%%%%%%%%%%%%%%%%%%%%%%%%%%%%%%%
%%%%%%%%%%%%%%%%%%%%%%%%%%%%%%%%%%%%%%%%%%%%%%%%%%%%%%%%%%%%%%%%%%%%%%%%%%%%%%%%%%%%%%%%%%%%%%%%%%%%%%%%%%%%%%%%%%%%%%%%%%%%%%%%


\thispagestyle{empty}%

{\raggedright%
Otto-Friedrich-Universität Bamberg\\% ggf. Namen der Universität eintragen
Lehrstuhl für Statistik und Ökonometrie\\%
Standort: Bamberg\\% eigenen Standort auswählen
Sommersemester 2022\\% Winter-/Sommersemester eintragen
Vorlesung: Blockseminar Survey-Methodik\\% Titel der Veranstaltung eintragen
Prüfer: Dr. Sara Bleniger\\% Name eintragen
}%

\vspace*{\fill}%
\begin{center}%
    \textbf{\Huge{Statistische Geheimhaltung:}}%
\end{center}%
\begin{center}%
    {\linespread{2.5}
    \textbf{\Huge{Cell Key Methode}}%
    }
\end{center}%
\vfill%

{\vspace*{\stretch{1}}%
\raggedleft%
Joshua Simon\\% eintragen
joshua-guenter.simon@stud.uni-bamberg.de\\% eintragen
Master Survey Statistik, 4. Fachsemester\\% eintragen
Matrikelnummer: 2032411\\%
\today\\}% fügt das aktuelle Datum ein, ggf. manuell ändern


\newpage%

%%%%%%%%%%%%%%%%%%%%%%%%%%%%%%%%%%%%%%%%%%%%%%%%%%%%%%%%%%%%%%%%%%%%%%%%%%%%%%%%%%%%%%%%%%%%%%%%%%%%%%%%%%%%%%%%%%%%%%%%%%%%%%%%% Angaben im Dokument ändern


%%%%%%%%%%%%%%%%%%%%%%%%%%%%%%%%%%%%%%%%%%%%%%%%%%%%%%%%%%%%%%%%%%%%%%%%%%%%%%%%%%%%%%%%%%%%%%%%%%%%%%%%%%%%%%%%%%%%%%%%%%%%%%%%

%%% Verzeichnisse

\tableofcontents% Inhaltsverzeichnis
\newpage%

\listoffigures% Abbildungsverzeichnis
\listoftables% Tabellenverzeichnis
\lstlistoflistings

\newpage%
%%%%%%%%%%%%%%%%%%%%%%%%%%%%%%%%%%%%%%%%%%%%%%%%%%%%%%%%%%%%%%%%%%%%%%%%%%%%%%%%%%%%%%%%%%%%%%%%%%%%%%%%%%%%%%%%%%%%%%%%%%%%%%%%

%%% Darstellender Teil

\onehalfspacing% Zeilenabstand


\section{Einführung}%

Die amtliche Statistik sorgt mit einer Vielzahl an Veröffentlichungen für die Bereitstellung von aufbereiteten statistischen Informationen. Damit geht sie dem Ziel nach, Bürgern, Institutionen und anderen gesellschaftlichen Einrichtungen eine Datengrundlage für die Entscheidungsfindung zu bieten. Weiter dient die amtliche Statistik auch der Politik und der Wissenschaft als Datenquelle. Das Sammeln und Erheben dieser Daten stellt in vielen Fällen einen Eingriff auf das Recht der informationellen Selbstbestimmung für Personen und Entitäten dar. Dieses Recht ist das Fundament des modernen Datenschutzes und wird über Artikel 2 des Grundgesetztes abgedeckt. Es steht au\ss er Frage, dass dieses Recht besonders schützenswert ist. Demnach steht auch die amtliche Statistik in der Pflicht dieser Verantwortung nachzukommen. Konkret manifestiert sich das Einhalten dieser Pflicht in dem sog. Statistikgeheimnis. Aus dem Bundestatistikgesetz lässt sich hierzu der folgende Absatz aufgreifen (\S 16 Abs. 1 Satz 1 BStatG):

\textit{\glqq Einzelangaben über persönliche und sachliche Verhältnisse, die für eine Bundesstatistik gemacht werden, sind von den Amtsträgern und für den öffentlichen Dienst besonders Verpflichteten, die mit der Durchführung von Bundesstatistiken betraut sind, geheim zu halten, soweit durch besondere Rechtsvorschrift nichts anderes bestimmt ist.\grqq{}}

Konkret möchte man mit der statistischen Geheimhaltung einen Schutz für einzelne Person und Entitäten vor der Offenlegung ihrer sensitiven Daten bieten. Dies dient im Weiteren auch der Aufrechterhaltung des Vertrauensverhältnisses zwischen den Befragten und den statistischen Ämtern und erhebenden Einrichtungen. Dies gewährleistet abermals die Zuverlässigkeit der Angaben und der Berichtswilligkeit der Befragten. Die vorausgegangenen Punkte werden in einer Begründung zum BStatG erwähnt [\cite{Nickl}]. Ausnahmen von einer Geheimhaltung bestehen nur in Ausnahmefällen, z.B. wenn eine explizite Einwilligung zur Veröffentlichung durch den Befragten vorliegt oder wenn sich die Informationen aus allgemein zugänglichen Quellen von öffentlichen Stellen beziehen. Auch die inner-behördlichen  Übermittlung, Methodenentwicklung, Planungs- und Forschungszwecke werden über das BStatG geregelt.

Im weiteren Verlauf dieser Arbeit sollen Geheimhaltungsverfahren und Geheimhaltungsregeln präsentiert werden, die im Einzelnen die statistische Geheimhaltung gewährleisten und damit dem Statistikgeheimnis der amtlichen Statistik nachkommen. Besondere Beachtung wird dabei der Cell Key Methode (CKM) geschenkt. Dieses Verfahren bietet ein Ansatz, welcher gegenüber anderen Verfahren gut zu implementieren und zu automatisieren ist. Gerade dieser Punkt ist in einem immer weiter werdenden technolgischen Umfeld nicht au\ss er Acht zu lassen.

%%%%%%%%%%%%%%%%%%%%%%%%%%%%%%%%%%%%%%%%%%%%%%%%%%%%%%%%%%%%%%%%%%%%%%%%%%%%%%%%%%%%%%%%%%%%%%%%%%%%%%%%%%%%%%%%%%%%%%%%%%%%%%%%%%%%%%%%%%%%

%\newpage
\section{Etablierte Geheimhaltungsverfahren}

In diesem Abschnitt sollen zunächst die grundlegenden Begrifflichkeiten für Geheimhaltungsverfahren und Geheimhaltungsregeln beschrieben werden. 

\subsection{Methodische Grundlagen}

Grundsätzlich gibt es zwei sich unterscheidende methodische Ansätze, die bei der Geheimhaltung zum Tragen kommen können [\cite{Nickl}]. Zum einen existieren \textit{informationsreduzierende Methoden}. In der Gattung dieser Verfahren werden durch Aggregation oder Sperrrung kritische Kategorien oder Werte die Aufdeckungsrisiken verhindert. Eine Aggregation meint in diesem Fall das Zusammenfassen zu übergeordneten Positionen, z.B. durch Summation kleinerer Positionen. Bei einer Sperrung ist auch oft von einer Löschung die Rede. Hier werden gezielt einzelne Werte identifiziert und aus der Tabelle entfernt. Als Kontrast stehen \textit{datenverändernde Methoden} gegenüber. Hier werden durch gezielte Veränderungen der Daten - beispielsweise durch Runden oder Zufallsüberlagerungen - kritische Werte verfälscht, was auch für eine erfolgreiche statistische Geheimhaltung sorgen kann. 

Weiter differenziert man Geheimhaltungsverfahren auch nach dem Zeitpunkt ihrer Durchführung. Die Daten können bereits vor der Tabellierung mit einer Geheimhaltung versehen werden. Man spricht hier von \textit{pre-tabulare Verfahren} [\cite{Rothe-1}]. Diese Verfahren werden als Anonymisierung bezeichnet, da die Daten im Vorfeld so verändert werden, dass keine kritischen Ergebnisse resultieren. Oftmals ist diese Art von Geheimhaltung aber nicht ausreichend, weshalb weitere Verfahren im Anschluss angewandt werden müssen. Man spricht nun von \textit{post-tabulare Verfahren} [\cite{Rothe-1}]. Ihre Mechanismen werden auf die fertig tabellierten Daten angewandt.


\subsection{Statistische Tabellen}

Gegenstand der Geheimhaltung stellen in dieser Arbeit statistische Tabellen dar. Ein Gro\ss teil der Veröffentlichungen der amtlichen Statistik sind selbst - oder beinhalten - statistische Tabellen, welche aus den amtlichen Daten abgeleitet werden. Ma\ss gebend für die Anwendung eines Geheimhaltungsverfahrens ist die Art der zu veröffentlichenden Tabelle, die vorliegt. Man unterscheidet im allgemeinen zwischen \textit{Häufigkeitstabellen} und \textit{Wertetabellen} [\cite{Nickl}]. Erstere stellen Häufigkeiten oder Fallzahlen dar, z.B. die Anzahl von Frauen und Männern innerhalb einer Universität. Wertetabellen hingegen stellen Wertesummen wie Umsätze dar. Demnach sind sie häufig in Wirtschaftsstatistiken anzutreffen [\cite{Rothe-2}]. Diese unterschiedlichen Kontexte, in denen die Zahlen dieser Tabellen interpretiert werden können, fordern eine natürliche Unterscheidung innerhalb der Geheimhaltung. Es folgen zwei einfache Beispiel für diese Tabellentypen mit rein fiktiven Ausprägungen.

\begin{table}[h]
    \centering
    \begin{tabular}{ r r r r }
        \textbf{Studienfach} \vline & \textbf{männlich} & \textbf{weiblich} & \textbf{insgesamt} \\ 
        \hline
        Bauingenieurwesen \vline & $4$ & $3$ & $7$ \\
        Informatik \vline & $9$ & $12$ & $21$ \\  
        Medizin \vline & $4$ & $1$ & $5$ \\
        Survey Statistik \vline & $10$ & $10$ & $20$ \\
        \hline
        Gesamt \vline & $27$ & $26$ & $53$
    \end{tabular}
    \caption{Beispiel für eine Häufigkeitstabelle}
\end{table}

\begin{table}[h]
    \centering
    \begin{tabular}{ r r r r r }
        \textbf{Brauerei} \vline & \textbf{Mährs Bräu} & \textbf{Schinkerla} & \textbf{Käsmann} & \textbf{Gesamt} \\ 
        \hline
        Umsatz \vline & $600.000$ & $50.000$ & $250.000$ & $900.000$
        \end{tabular}
    \caption{Beispiel für eine Wertetabelle}
\end{table}


\subsection{Bewährte Ansätze}

Nachdem nun die grundlegenden Begriffe im Zusammenhang mit der Geheimhaltung von statistischen Tabellen erläutert wurden, beschäftigt sich dieser Abschnitt mit bewährten Ansätzen, um diese Geheimhaltung durchzuführen. Zunächst werden die Mechanismen für eine pre-tabulare Geheimhaltung - also eine Anonymisierung - beschrieben. Gegenstand der Betrachtung sind hier immer die Einzeldatensätze. Es existieren verschiedene Stufen der Anonymität. Diese werden in [\cite{Rothe-1}] wie folgt dargestellt:
\begin{itemize}
    \item \textbf{formale Anonymisierung:} Diese Art der Anonymisierung wird intern in den statistischen Landesämtern verwendet. Durch Entfernung aller direkten Identifikatoren [\cite{Nickl}], wie z.B. Name, Adresse oder Matrikelnummer, bleibt der eigentliche Informationsgehalt für die statistischen Auswertungen hoch. 
    \item \textbf{faktische Anonymisierung:} Für die unabhängige wissenschaftliche Forschung werden die Daten so bearbeitet, dass unter realistischen Rahmenbedingungen keine erfolgreiche Identifikation möglich ist [\cite{Nickl}].
    \item \textbf{absolute Anonymisierung:} Diese Variante bietet den höchsten Informationsverlust, stellt im Gegenzug aber sicher, dass unter keinen Umständen eine Re-Identifikation  vorgenommen werden kann [\cite{Nickl}]. Damit ist dieser Ansatz zu wählen, wenn die Tabellen an die breite Öffentlichkeit weitergeben werden sollen.
\end{itemize}
Bei allen drei Optionen ist das darunterliegende Vorgehen identisch. Im ersten Schritt wird immer die jeweilige Anonymisierung auf die Einzeldatensätze angewandt. Anschlie\ss end werden die statistischen Tabellen generiert. Für weitere Einzelheiten sei an dieser Stelle auf [\cite{Rothe-1}] verwiesen.

Im Weiteren wird speziell auf die post-tabularen Verfahren für Häufigkeits- und Wertetabelle eingegangen. Am Anfangs eines jeden Geheimhaltungsverfahrens stehen die sog. \textit{Geheimhaltungsregel}, die eine Identifizierung kritischer Fälle erlauben. Eine Standardansatz in der amtlichen Statistik stellt hier die \textit{Mindestfallzahlregel} [\cite{Rothe-1}] dar. Diese Regel lässt sich auf Häufigkeitstabellen anwenden, indem die kritischen Fälle mit einem zuvor festgelegten Wert $n$ verglichen werden. Meist wird hier $n = 3$ gewählt [\cite{Rothe-1}]. Ist der betrachtete Tabellenwert kleiner als der Wert $n$, so ist dieser Tabellenwert geheim zu halten. Für Wertetabellen hingegen lässt sich beispielsweise die \textit{p-\% Regel} anwenden [\cite{Rothe-2}]. Dieser Ansatz besagt, dass ein Tabellenwert $x$ geheim zu halten ist, wenn 

\begin{align}
    x - x_2 - x_1 < \frac{p}{100} \cdot x_1
\end{align}

gilt [\cite{Rothe-2}], wobei in $(1)$ $x_1$ der grö\ss te und $x_2$ der zweitgrö\ss te Beitrag ist. In anderen Worten lässt sich sagen, dass der Tabellenwert $x$ genau dann geheim zu halten ist, wenn die Differenz zwischen $x$ und $x_1$, $x_2$ nicht mindestens $p\%$ vom grö\ss ten Beitrag $x_1$ beträgt.

Um letztlich auch die Geheimhaltung durchzuführen wird unter diesen Regeln oft auf das Verfahren der \textit{Zellsperrung} zurückgegriffen [\cite{Rothe-1}]. Hierbei werden die in einem ersten Schritt mit der Geheimhaltungsregel identifizierten Fälle durch einen Punkt $\cdot$ ersetzt. Bei dieser Sperrung der kritischen Werte spricht man von der \textit{Primärsperrung}. Um aber auch Rückrechnungen unter Zuhilfenahme von Rand- oder Zwischensummen zu verhindern, wird eine \textit{Sekundärsperrung} angewandt. Diese sperrt neben dem bereits durch die Primärsperrung unkenntlichen gemachten Zellwerts auch jeweils eine Zellwert in derselben Zeile, in derselben Spalte sowie dasjenige Tabellenfeld, in dem die Zeile und die Spalte der beiden zuvor genannten Felder aufeinander treffen [\cite{Rothe-1}]. Es wird schnell klar, dass unter Anwendung dieses Verfahrens ein hoher Informationsverlust zu Gunsten der Geheimhaltung entstehen kann. Um dieses Verlust möglichst gering zu halten, sollten eher Tabelleninnenfelder und keine Summenfelder gesperrt werden [\cite{Rothe-1}]. Die Zellsperrung stellt nach Anwendung ein sicheres Verfahren dar, welches allerdings durch die oftmals mit hohem manuellem Aufwand verbundene Sekundärsperrung unflexibel ist. Für ein anschauliches Beispiel sei an dieser Stelle auf [\cite{Rothe-1}] verwiesen.



%%%%%%%%%%%%%%%%%%%%%%%%%%%%%%%%%%%%%%%%%%%%%%%%%%%%%%%%%%%%%%%%%%%%%%%%%%%%%%%%%%%%%%%%%%%%%%%%%%%%%%%%%%%%%%%%%%%%%%%%%%%%%%%%%%%%%%%%%%%%


\section{Cell Key Methode}

Die im vorherigen Kapitel beschrieben Geheimhaltungsverfahren müssen in der Regel - zumindest bis zu einem gewissen Grad - manuell durchgeführt werden und eine Automatisierung ist demnach schwer umzusetzen oder mit vielen Kompromissen verbunden. Mit der \textit{Cell Key Methode (CKM)} wird ein Geheimhaltungsverfahren präsentiert, welches gut zu automatisieren und vergleichsweise einfach zu implementieren ist. Die Cell Key Methode ist auch als \textit{ABS-Verfahren} bekannt. Der Name stammt von der schöpfenden Institution des Verfahrens, dem Australian Bureau of Statistics, ab. Durch die Verwendung von zufallsbasierten Additionen, den sog. Überlagerungen, werden Datenwerte verschleiert. Der Ermittlung einer solchen zufallsbasierten Addition liegt eine einmalig festzulegende Wahrscheinlichkeitsverteilung mit den möglichen Überlagerungswerten zugrunde [\cite{Enderle}]. Für die Bestimmung dieser Überlagerungen wird ein deterministischer Mechanismus eingesetzt. Dieser nutzt den original Zellwert und den sog. Cell-Key um aus der Verteilung der Überlagerungswerte eine eindeutige Überlagerung zu ziehen [\cite{Enderle}]. Eine mit dem CKM Verfahren geheim gehaltene  Tabelle veröffentlicht also die Summe aus Originalwerten und Überlagerungen. Die CKM zählt damit zu den datenverändernden Verfahren.

\subsection{Verfahrensparameter und Überlagerungsmatrix}

An statistische Geheimhaltungsverfahren, insbesondere den datenverändernden Verfahren, werde bestimmte Anforderungen gestellt. Für die Cell Key Methode werden in der amtlichen Statistik gewisse stochastische Eigenschaften gefordert, um die Qualität und Nachvollziehbarkeit der Ergebnisse zu sichern. Zu diesen Eigenschaften zählt einerseits die Unverzerrtheit der Überlagerungen [\cite{Enderle}]. Damit meint man, dass der Überlagerungswert, welcher zu den Originalwerten addiert wird, im Mittel gleich Null ist. Es soll also die Erwartungstreue $E(z) = 0$ gelten. Darüber hinaus fodert man ebenfalls eine konstante Streuung der Verteilung der Überlagerungen [\cite{Enderle}]. Es soll die Varianz erhalten bleiben, also $Var(z) = s^2$ gelten. Um diese beiden Eigenschaften zu konkretisieren, werden Verfahrensparameter eingeführt. Diese dienen weiter auch dazu das Geheimhaltungsverfahren - und damit die Überlagerungen - an verschiedene Kontexte anzupassen. Zu diesen Verfahrensparameter zählen nach [\cite{Höhne}]:

\begin{itemize}
    \item Eine boolesche Variable, die angibt, ob Originalwerte $1$ und $2$ geheim gehalten werden sollen.
    \item Der Anteil $p_0$ der nicht zu überlagernden Originalwerte.
    \item Die Maximalüberlagerung $d$.
    \item Die Standardabweichung der Überlagerungsbeiträge $s$.
\end{itemize}

Gegenstand des Interesses sind demnach Zufallsfunktionen $z$, die eine bedingte Wahrscheinlichkeitsverteilung auf die Originalwerte $i$ mit Zielhäufigkeit $j$ darstellen. Man sucht also für jeden Originalwert $i = \{0, 1, 2, \dots, n \}$ eine Wahrscheinlichkeitsverteilung der Form $z = p_i$ mit den Wahrscheinlichkeiten für die Übergänge $v_i$ hin zu den Zielhäufigkeiten $j$ [\cite{Enderle}]. Diese Wahrscheinlichkeiten bilden ein nicht-lineares Gleichungssystem, in welchem die zuvor genannten stochastischen Eigenschaften als Nebenbedingungen eingehen. Diese lassen sich nun als 

\begin{align}
    E(z) & = \sum_{i = -d}^{d} p_i v_i = 0 \\
    Var(z) & = \sum_{i = -d}^{d} p_i v_{i}^{2} = s^2 \\
    & \sum_{i = -d}^{d} p_i = 1
\end{align}

schreiben [\cite{Höhne}]. Dabei wird in der letzten Zeile $(4)$ noch gefordert, dass die Summe der Übergangswahrscheinlichkeiten für einen Originalwert gleich $1$ ist. Die Lösung dieses Problems lässt sich in Matrixform notieren. Man spricht hier von der sog. Überlagerungsmatrix mit Zeilen $i$ und Spalten $j$, welche zur Bestimmung der Überlagerungsbeiträge $j$ zu den Originalwerten $i$ verwendet werden kann. Für weitere mathematische Details und Lösungsansätze sei an dieser Stelle auf [\cite{Giessing}] verwiesen. In [\cite{Höhne}] wird die Lösung eines solchen Gleichungssystems für die Verfahrensparameter $p_0 = 0,5 = 50\%$, $d = 4$ und $s = 2,25$ gezeigt. Diese Überlagerungsmatrix ist in Abbildung \ref{matrix_plot} in Form einer Heatmap zu sehen. Der \textit{Python} \footnote{Python ist eine universelle Programmiersprache, die sich einer hohen Beliebtheit im Bereich Data Science und künstlicher Intelligenz erfreut.} Code zum Erstellen dieser Grafik ist in Anhang A zu finden.

\begin{figure}[H]
    \begin{center}
        \includegraphics[width=0.85\textwidth]{img/matrix.png}
        \caption{Beispiel für eine Überlagerungsmatrix mit Übergangswahrscheinlichkeiten für paarige Originalwerte und Zielhäufigkeiten}
        \label{matrix_plot}
    \end{center}
\end{figure}

Es ist hierbei anzumerken, dass direkt eine symmetrische Verteilung für die Überlagerungsbeiträge gewählt wurde. Auf der $x$-Achse sind die anzuwendenden Überlagerungswerte von $-d$ bis $d$ zu sehen. Auf der $y$-Achse sind die Originalwerte abgetragen. In der jeweiligen Zelle sind die einzelnen Übergangswahrscheinlichkeiten abzulesen. Für Originalwerte grö\ss er $7$ ist die Zeile für Originalwerte gleich $7$ ma\ss gebend. Die Überlagerungsmatrix bildet den Kern der CKM. Ihre Anwendung wird im nächsten Abschnitt beschrieben.

Es gibt bereits erste Open Source Softwarelösungen, die die Funktionalitäten zum Berechnen einer solchen Überlagerungsmatrix bereitstellen. Wie in [\cite{Giessing-2}] beschrieben, kann hierfür beispielsweise das \textit{R} \footnote{R ist eine statistik-nahe Programmiersprache.} Paket \textit{ptable} \footnote{Verfügbar unter \url{https://github.com/tenderle/ptable/}.} herangezogen werden. 


\subsection{Methodik und Verfahrensdurchführung}%

Die wichtigsten Bestandteile des Verfahrens werden in [\cite{Enderle}] dargestellt. Ähnlich wie in [\cite{Wipke}] beschrieben, lässt sich nun ein Algorithmus formulieren. Ausgangspunkt sind die in einer Statistik erhobenen Mikrodaten bzw. Mikrodatensätze. Das sind die plausibilisierten Einzeldatensätze. 

\begin{enumerate}
    \item Erzeugung der Originalwerte mit einem Auswertungstool
    \item Cell-Key-Bestimmung aus Zufallszahlen innerhalb des Auswertungs-Tools
    \item Lookup-Modul
    \begin{enumerate}
        \item Auslesen der Überlagerungswerte aus der Überlagerungsmatrix
        \item Addieren der Überlagerungswerte und Originalwerte
    \end{enumerate}
\end{enumerate}

Die folgende Abbildung \ref{fig_ckm_chart} visualisiert den schematischen Ablauf des zuvor beschrieben Algorithmus. Im Wesentlichen werden zwei technische System benötigt, um das Verfahren zu realisieren. Zum einen wird ein Auswertungstool benötigt, welches die gespeicherten Mikrodaten in Tabellenform bringt. Mit Hilfe des Lookup-Moduls werden dann die beiden Eingangsgrö\ss en - Originalwerte und Cell-Keys - verwendet, um die Überlagerungswerte aus der Überlagerungsmatrix zu bestimmen. Diese Überlagerungswerte werden letztlich auf die Originalwerte addiert und stellen damit die finalen Werte für die Veröffentlichung dar. 

\begin{figure}[H]
    \begin{center}
        \includegraphics[width=0.85\textwidth]{img/ckm_flow.png}
        \caption{Ablaufdiagramm der Cell Key Methode}
        \label{fig_ckm_chart}
    \end{center}
\end{figure}

Die Einzelschritte des Verfahrens sollen nun im Detail beleuchtet werden.

\subsubsection{Erzeugung der Originalwerte}

Dieser Schritt ist spezifisch für die jeweilige Zieltabelle, die veröffentlicht werden soll. Im Allgemeinen werden Filterungen anhand bestimmter Merkmale vorgenommen und dann Summen der Fallzahlen gebildet. Für diese Operation - also die Tabellierung - führt man die Bezeichnung $f$ ein.

\subsubsection{Cell-Key-Bestimmung}

Für die Bestimmung des Cell-Keys wird jedem Mikrodatensatz zunächst eine gleichverteile Zufallszahl $r$, der sog. \textit{Record-Key}, mit $r \sim \mathcal{U}(0, 1)$ zugeordnet. Mit diesen Record-Keys wird dieselbe Auswertungstabelle wie mit den Originalwerten gebildet. Man führt also dieselbe Operation $f$ auf den Daten durch, wie in Abschnitt 3.1.1. Es ergeben sich also Summen von Record-Keys. Von diesen Record-Key Summen werden nur die Nachkommastellen betrachtet. Dieser Wert definiert den \textit{Cell-Key} [\cite{Enderle}]. Um dieses Vorgehen zu verdeutlichen soll das folgende Beispiel aus der Hochschulstatistik betrachtet werden. Tabelle \ref{tab_mikrodaten} zeigt die Mikrodaten mit Record-Keys, wie sie in einer Datenbank gespeichert sein könnten.

\begin{table}[h]
    \centering
    \begin{tabular}{ r r r r }
        \textbf{ID} \vline & \textbf{Universität} & \textbf{Geschlecht} & \textbf{Record-Key} \\ 
        \hline
        $1$ \vline & Würzburg & m & $0,611853$ \\
        $2$ \vline & Eichstätt & w & $0,139494$ \\
        $3$ \vline & München & w & $0,292145$ \\
        $4$ \vline & München & m & $0,366362$ \\
        $5$ \vline & Würzburg & m & $0,456070$ \\
        $6$ \vline & Würzburg & m & $0,785176$ \\
        $7$ \vline & Bamberg & m & $0,199674$ \\
        $8$ \vline & München & w & $0,514234$ \\
        $9$ \vline & München & m & $0,592415$ \\
        $10$ \vline & München & m & $0,046450$
    \end{tabular}
    \caption{Mikrodaten mit Record-Keys}
    \label{tab_mikrodaten}
\end{table}

Tabelle \ref{tab_agg} stellt die Daten nach Durchführung der Tabellierung $f$ dar. Die Daten wurden hier nach der Universität und nach dem Geschlecht zusammengefasst. Die entsprechenden Fallzahlen, Record-Key-Summen und Cell-Keys werden abgebildet. 

\begin{table}[h]
    \centering
    \begin{tabular}{ r r r r r r}
        \textbf{ID} \vline & \textbf{Universität} & \textbf{Geschl.} & \textbf{Fallzahl} & \textbf{Record-Key-Summe} & \textbf{Cell-Key} \\ 
        \hline
        $1$ \vline & Bamberg & m & $1$ & $0,199674$ & $0,199674$ \\
        $2$ \vline & Eichstätt & w & $1$ & $0,139494$ & $0,139494$ \\
        $3$ \vline & München & m & $3$ & $1,005227$ & $0,005227$ \\
        $4$ \vline & München & w & $2$ & $0,806379$ & $0,806379$ \\
        $5$ \vline & Würzburg & m & $3$ & $1,853099$ & $0,853099$
    \end{tabular}
    \caption{Aggregierte Daten mit Record-Key-Summen und Cell-Key}
    \label{tab_agg}
\end{table}

Die Werte in Zeile 5 von Tablle \ref{tab_agg} ergeben sich durch Zählen der Zeilen in Tabelle \ref{tab_mikrodaten}, in denen $\left( Universit\ddot{a}t = W\ddot{u}rzburg \: \wedge \: Geschlecht = m \right)$. Die Record-Key-Summe in dieser Zeile ergibt sich nach $0,611853 + 0,456070 + 0,785176 = 1,853099$. Der daraus abgeleitete Cell-Key beträgt damit $0,853099$.

\subsubsection{Lookup-Modul}

Im Anschluss an die Bestimmung der Originalwerte und der dazugehörigen Cell-Keys gilt es nun die Überlagerungen zu berechnen. Für die Bestimmung eines Überlagerungswertes dient das Paar $(Originalwert, \; CellKey)$ als Input. Damit meint man den Wert einer einzelnen Tabellenzelle und den über dieselbe Operation $f$ berechneten Cell-Key. Das Lookup Modul stellt die Funktionalität bereit, anhand dieses Wertepaares den zugehörigen Überlagerungswert aus der Überlagerungsmatrix abzulesen. Um dies weiter zu illustrieren wird die Überlagerungsmatrix aus Abbildung \ref{matrix_plot} herangezogen. Betrachtet man Zeile 5 aus Tabelle \ref{tab_agg}, so liegt das Paar $(Originalwert = 3, \; CellKey = 0,853099)$ vor. Um nun auch den dazu passenden Überlagerungswert zu bestimmen, wählt man die Zeile $i$ der Überlagerungsmatrix, die dem $Originalwert$ - hier also $3$ - entspricht. Anschlie\ss end bestimmt man mit Hilfe des $CellKey$ die Spalte $j$, für welche der Wert in der Matirx erstmalig kleiner als $CellKey$ ist. In diesem Fall führt dies zu $j = 1$, da $0,853099 < 0.86$. Der Wert $j = 1$ ist damit der zu addierende Überlagerungswert. Daraus ergibt sich ein finaler Wert von $3 + 1 = 4$. Wendet man dieses Vorgehen vollständig auf das Beispiel aus Tabelle \ref{tab_agg} an, so ergeben sich die finalen und damit geheim gehaltene n Werte aus Tabelle \ref{tab_final}.

\begin{table}[h]
    \centering
    \begin{tabular}{ r r r r r r}
        \textbf{ID} \vline & \textbf{Universität} & \textbf{Geschl.} & \textbf{Fallzahl} & \textbf{Überlagerung} & \textbf{Finaler Wert} \\ 
        \hline
        $1$ \vline & Bamberg & m & $1$ & $-1$ & $0$ \\
        $2$ \vline & Eichstätt & w & $1$ & $-1$ & $0$ \\
        $3$ \vline & München & m & $3$ & $-3$ & $0$ \\
        $4$ \vline & München & w & $2$ & $1$ & $3$ \\
        $5$ \vline & Würzburg & m & $3$ & $1$ & $4$
    \end{tabular}
    \caption{Überlagerungen und final geheim gehaltene  Werte}
    \label{tab_final}
\end{table}

In Anhang A ist eine \textit{Python} Realisierung dieses Verfahrens zu finden.


\subsection{Besonderheiten der Cell Key Methode}%

Eine Besonderheit, die sich aus der Verwendung der Cell Key Methode ergibt, ist die Nicht-Additivität des Verfahrens. Dadurch, dass Rand- und Zwischensummen nicht erst nach der Geheimhaltung, sondern ebenfalls während der Tabellierung gebildet werden, unterliegen auch diese Werte dem Geheimhaltungsmechanismus. Um dies zu veranschaulichen, soll abschlie\ss end ein weiteres Fallbeispiel aus der Hochschulstatistik betrachtet werden. Tabelle \ref{tab_additivity} zeigt für eine Universität jeweils eine Insgesamt-Position i sowie die Unterteilung nach dem Geschlecht m und w. Es sind sowohl die Originalwerte als auch die final überlagerten Werte abgebildet. Auch die Positionen für die Zwischensummen i wurden mit der Tabellierungs-Operation $f$ auf Basis der Originalwerte und der Record-Keys gebildet. Damit entstehten also ein eigner Cell-Key und Überlagerungswert für diese Zahlen. Für das Beispiel der Universität Bamberg aus Tabelle \ref{tab_additivity} sieht man schnell, dass für die Originalwerte die gewohnte Additivität vorhanden ist, denn $166 = 75 + 91$. Führt man diese Betrachtung allerdings auf den überlagerten Werten durch, so erhält man eine falsche Aussage, denn $168 \neq 75 + 91$. Diese Phänomen bezeichnet man als \textit{Nicht-Additivität} einer statistischen Tabelle.

\begin{table}[h]
    \centering
    \begin{tabular}{ r r r r r}
        \textbf{ID} \vline & \textbf{Universität} & \textbf{Geschl.} & \textbf{orig. Fallzahl} & \textbf{überl. Fallzahl} \\ 
        \hline
        $1$ \vline & Bamberg & i & $166$ & $168$ \\
        $2$ \vline & Bamberg & m & $75$ & $75$ \\
        $3$ \vline & Bamberg & w & $91$ & $91$ \\
        $4$ \vline & Eichstätt & i & $46$ & $48$ \\
        $5$ \vline & Eichstätt & m & $17$ & $20$ \\
        $6$ \vline & Eichstätt & w & $32$ & $29$
    \end{tabular}
    \caption{Beispiel zur Nicht-Additivität der CKM}
    \label{tab_additivity}
\end{table}

Diese Nicht-Additivität wird beim Cell Key Verfahren in Kauf genommen, da im Weiteren zwei konkrete Vorteile dieser Methode überwiegen. Das unabhängige und separate Überlagern von Tabellenfeldern führt zu diesen wichtigen Vorteilen dieses Geheimhaltungsverfahrens [\cite{Enderle}]. Zum einen wird eine \textit{tabellenübergreifende Konsistenz} erreicht. Die zu addierenden Überlagerungswerte zu einem bestimmten Originalwert (z.B. Studierende der Universität Bamberg) sind bei gleicher Datengrundlage immer identisch - unabhängig von der Zieltabelle. Dies ergibt sich aus den einmalig zugespielten Record-Keys und dem danach folgenden deterministischen Lookup-Modul. Der zweite überwiegende Vorteil der CKM ist die \textit{Genauigkeit} des Verfahrens [\cite{Enderle}]. Dieser Punkt beschreibt die Vermeidung von zufällig gleichgerichteten Überlagerungen. Dies würde im Einzelfall zu grö\ss eren Veränderungen zwischen Original- und geheim gehaltene n Wert führen. Durch die Überlagerung aller Tabellenzellen umgeht man diese Gefahr.


\subsection{Aufdeckungsrisiko}

Damit die Cell Key Methode ein sicheres Geheimhaltungsverfahren bleibt, dürfen die Verfahrensparameter unter keinen Umständen der Öffentlichkeit preisgegeben werden. Sind beispielsweise die Maximalabweichung oder die Standardabweichung der Überlagerungen bekannt, so können in Tabellenfeldern, die einen additiven Zusammenhang aufweisen, Rückschlüsse auf die Originalwerte mit Hilfe der Randsummen gezogen werden. Für Details sei hier auf [\cite{Höhne}] und [\cite{Giessing}] verwiesen.

%%%%%%%%%%%%%%%%%%%%%%%%%%%%%%%%%%%%%%%%%%%%%%%%%%%%%%%%%%%%%%%%%%%%%%%%%%%%%%%%%%%%%%%%%%%%%%%%%%%%%%%%%%%%%%%%%%%%%%%%%%%%%%%%%%%%%%%%%%%%

%\newpage
\section{Fazit}

Mit der Cell Key Methode wurde ein modernes Geheimhaltungsverfahren präsentiert, welches sich durch seine Flexibilität - auch in der technischen Integration - von bestehenden Geheimhaltungsverfahren unterscheidet. So ist es ein hochattraktives Verfahren zur Gewährleistung der Geheimhaltung in neuen technischen Systemen, wie der Auswertungsdatenbank der amtlichen Hochschulstatistik, oder auch dem anstehenden Zenus. Mit dem Wegfallen von manuellen Geheimhaltungsprozeduren lassen sich qualitativ hochwertige Ergenisse in deutlich kürzerer Zeit und mit übergreifender Konsistenz veröffentlichen. Allerdings ist auch die Besonderheit der Nicht-Additivität deutlich an die Endnutzer der Daten zu kommunizieren, um eine Missinterpretation der Ergebnisse zu verhindern. Durch die wählbaren Verfahrensparameter wird das CKM-Verfahren zukünftig sicher auch in anderen Teilen der amtlichen Statistik Einzug finden.

\newpage%
%%%%%%%%%%%%%%%%%%%%%%%%%%%%%%%%%%%%%%%%%%%%%%%%%%%%%%%%%%%%%%%%%%%%%%%%%%%%%%%%%%%%%%%%%%%%%%%%%%%%%%%%%%%%%%%%%%%%%%%%%%%%%%%%
%%%%%%%%%%%%%%%%%%%%%%%%%%%%%%%%%%%%%%%%%%%%%%%%%%%%%%%%%%%%%%%%%%%%%%%%%%%%%%%%%%%%%%%%%%%%%%%%%%%%%%%%%%%%%%%%%%%%%%%%%%%%%%%%

%%% Literaturverzeichnis
%%%%%%%%%%%%%%%%%%%%%%%%%%%%%%%%%%%%%%%%%%%%%%%%%%%%%%%%%%%%%%%%%%%%%%%%%%%%%%%%%%%%%%%%%%%%%%%%%%%%%%%%%%%%%%%%%%%%%%%%%%%%%%%%
%%%%%%%%%%%%%%%%%%%%%%%%%%%%%%%%%% Vorlage für Statistik-Hausarbeiten: Literaturverzeichnis %%%%%%%%%%%%%%%%%%%%%%%%%%%%%%%%%%%%
%%%%%%%%%%%%%%%%%%%%%%%%%%%%%%%%%%%%%%%%%%%%%%%%%%%%%%%%%%%%%%%%%%%%%%%%%%%%%%%%%%%%%%%%%%%%%%%%%%%%%%%%%%%%%%%%%%%%%%%%%%%%%%%%


\begin{thebibliography}{9}
    % beliebige Zahl steht für die Anzahl der Quellen, z.B. 10 Quellen - zweistellige Zahl, 69 Quellen - zweistellige Zahl, 157 Quellen - dreistellige Zahl
\singlespacing%

\bibitem[LfStat]{LfStat}%
Bayerisches Landesamt für Statistik (o.\,J.): \emph{Hochschulen}, online verfügbar unter: \url{https://www.statistik.bayern.de/statistik/bildung_soziales/hochschulen/index.html#link_1} (Zugriff am 29.12.2020).%

\bibitem[Cri20]{cri20}%
Crickard, Paul (2020): \emph{Data Engineering with Python}. Birmingham: Packt Publishing.%

\bibitem[Kim13]{kim13}%
Kimball, Ralph and Ross, Margy (2013): \emph{The Data Warehouse Toolkit: The Definitive Guide to Dimensional Modeling}. Indianapolis: Wiley.%

\bibitem[HStatG]{HStatG}
Statistisches Bundesamt (o.\,J.): \emph{Gesetz über die Statistik für das Hochschulwesen (Hochschulstatistikgesetz – HStatG)}, online verfügbar unter: \url{https://www.destatis.de/DE/Methoden/Rechtsgrundlagen/Statistikbereiche/Inhalte/505_HStatG.pdf?__blob=publicationFile} (Zugriff am 29.12.2020).

\bibitem[Wik19]{T-SQL}%
Wikipedia (2019): \emph{T-SQL - Wikipedia, Die freie Enzyklopädie}, online verfügbar unter: \url{https://de.wikipedia.org/wiki/Transact-SQL} (Zugriff am 10.01.2021).%

\bibitem[Wik21]{Datawarehouse}%
Wikipedia (2021): \emph{Datawarehouse - Wikipedia, Die freie Enzyklopädie}, online verfügbar unter: \url{https://de.wikipedia.org/wiki/Data_Warehouse} (Zugriff am 10.01.2021).%

\bibitem[Wik21]{ETL}%
Wikipedia (2021): \emph{ETL-Prozess - Wikipedia, Die freie Enzyklopädie}, online verfügbar unter: \url{https://de.wikipedia.org/wiki/ETL-Prozess} (Zugriff am 10.01.2021).%


%\bibitem[Name, wie er mit cite im Text zitiert wird]{kuerzel ohne Umlaute, das mit cite angesprochen wird}%
%Nachname, Vorname (Jahr): \emph{Buchtitel}. Ort: Verlag.%

%\bibitem[Autor 2013]{mimimi}%
%Nachname, Vorname (Jahr): Artikel, in: \emph{Zeitschrift} Auflage, S. XX-YY.%

%\bibitem[Autor et\,al. 1999]{magnicht}%
%Nachname, Vorname (o.\,J.): Titel, online verfügbar unter: \url{http...} (Zugriff am \today).%

\end{thebibliography}%

\newpage%

%%%%%%%%%%%%%%%%%%%%%%%%%%%%%%%%%%%%%%%%%%%%%%%%%%%%%%%%%%%%%%%%%%%%%%%%%%%%%%%%%%%%%%%%%%%%%%%%%%%%%%%%%%%%%%%%%%%%%%%%%%%%%%%%%

\newpage 
\appendix
\section{Python Implementierung}

Nachfolgend ist die vollständige \textit{Python} Implementierung des CKM Verfahrens für ein Testbeispiel aus der Hochschulstatistik abgebildet. Für die Werte der Überlagerungsmatrix wurde auf die Literatur zurückgegriffen.

\lstinputlisting[language=Python, caption=CKM Python Beispiel]{../src/ckm_paper.py}

%%%%%%%%%%%%%%%%%%%%%%%%%%%%%%%%%%%%%%%%%%%%%%%%%%%%%%%%%%%%%%%%%%%%%%%%%%%%%%%%%%%%%%%%%%%%%%%%%%%%%%%%%%%%%%%%%%%%%%%%%%%%%%%%
%%%%%%%%%%%%%%%%%%%%%%%%%%%%%%%%%%%%%%%%%%%%%%%%%%%%%%%%%%%%%%%%%%%%%%%%%%%%%%%%%%%%%%%%%%%%%%%%%%%%%%%%%%%%%%%%%%%%%%%%%%%%%%%%

%%% Wahrheitsgemäße Erklärung

\newpage 
\noindent%
Ich erkläre hiermit, dass ich die Seminararbeit mit dem Titel \emph{Statistische Geheimhaltung: Cell Key Methode} im \emph{Sommersemester 2022} selbständig angefertigt, keine anderen Hilfsmittel als die im Literaturverzeichnis genannten benutzt und alle aus den Quellen und der Literatur wörtlich oder sinngemä\ss übernommenen Stellen als solche gekennzeichnet habe.%
\bigskip
 
\noindent%
\emph{Bamberg}, den \today\\%
\emph{Unterschrift}%

\end{document}%

%%%%%%%%%%%%%%%%%%%%%%%%%%%%%%%%%%%%%%%%%%%%%%%%%%%%%%%%%%%%%%%%%%%%%%%%%%%%%%%%%%%%%%%%%%%%%%%%%%%%%%%%%%%%%%%%%%%%%%%%%%%%%%%%