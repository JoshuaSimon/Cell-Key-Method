%%%%%%%%%%%%%%%%%%%%%%%%%%%%%%%%%%%%%%%%%%%%%%%%%%%%%%%%%%%%%%%%%%%%%%%%%%%%%%%%%%%%%%%%%%%%%%%%%%%%%%%%%%%%%%%%%%%%%%%%%%%%%%%%
%%%%%%%%%%%%%%%%%%%%%%%%%%%%%%%%%%%%%%%% Vorlage für Statistik-Hausarbeiten: Kopfdatei %%%%%%%%%%%%%%%%%%%%%%%%%%%%%%%%%%%%%%%%%
%%%%%%%%%%%%%%%%%%%%%%%%%%%%%%%%%%%%%%%%%%%%%%%%%%%%%%%%%%%%%%%%%%%%%%%%%%%%%%%%%%%%%%%%%%%%%%%%%%%%%%%%%%%%%%%%%%%%%%%%%%%%%%%%


%%%%%%%%%%%%%%%%%%%%%%%%%%%%%%%%%%%%%%%%%%%%%%%%%%%%%%%%%%%%%%%%%%%%%%%%%%%%%%%%%%%%%%%%%%%%%%%%%%%%%%%%%%%%%%%%%%%%%%%%%%%%%%%%
%%%%%%%%%%%%%%%%%%%%%%%%%%%%%%%%%%%%%%%%% Vorlage für Statistik-Hausarbeiten: Header %%%%%%%%%%%%%%%%%%%%%%%%%%%%%%%%%%%%%%%%%%%
%%%%%%%%%%%%%%%%%%%%%%%%%%%%%%%%%%%%%%%%%%%%%%%%%%%%%%%%%%%%%%%%%%%%%%%%%%%%%%%%%%%%%%%%%%%%%%%%%%%%%%%%%%%%%%%%%%%%%%%%%%%%%%%%


%%% Basics

\documentclass[bibtotoc, 12pt, numbers=endperiod, openbib]{scrartcl}%
\usepackage[a4paper, left=30mm, right=30mm, top=20mm, bottom=20mm, includefoot]{geometry}% Seitenränder
\usepackage[utf8]{inputenc}% in TeXnicCenter bitte umändern in: \usepackage[latin1]{inputenc}
\usepackage[T1]{fontenc}%
\usepackage[ngerman]{babel}% in englischsprachigen Arbeiten: \usepackage[english]{babel}
\usepackage{mathptmx}%
\usepackage{amsmath, amsfonts, amssymb}%
\usepackage{graphicx}%


%%%%%%%%%%%%%%%%%%%%%%%%%%%%%%%%%%%%%%%%%%%%%%%%%%%%%%%%%%%%%%%%%%%%%%%%%%%%%%%%%%%%%%%%%%%%%%%%%%%%%%%%%%%%%%%%%%%%%%%%%%%%%%%%

%%% Aussehen allgemein

\usepackage{lmodern}% Schriftart lmodern
\usepackage{setspace}% Zeilenabstand
\pagenumbering{arabic}% arabische Seitenzahlen
\setkomafont{sectioning}{\bfseries}% Serifenschrift für Überschriften
%\usepackage[small]{titlesec}%% verkleinert die Überschriften etwas
%\usepackage[singlelinecheck=0]{caption}% linksbündige Abbildungsüberschriften


%%%%%%%%%%%%%%%%%%%%%%%%%%%%%%%%%%%%%%%%%%%%%%%%%%%%%%%%%%%%%%%%%%%%%%%%%%%%%%%%%%%%%%%%%%%%%%%%%%%%%%%%%%%%%%%%%%%%%%%%%%%%%%%%

%%% Zitate und Literaturverzeichnis

% Zitate mittels \cite[Seitenzahl]{kuerzel}
\usepackage[noadjust]{cite}%
\renewcommand\citeleft{}% Zeichen links vom Zitat, z.B. (, ggf. in die Klammer einfügen
\renewcommand\citeright{}% Zeichen rechts vom Zitat, z.B. ), ggf. in die Klammer einfügen
\renewcommand\citemid{:  }% nach der Seitenzahl folgt ": ", nach Belieben veränderbar

% Literaturverzeichnis
\makeatletter%
\renewcommand{\@biblabel}[1]{}% Literaturverzeichnis nicht nummeriert, keine Klammern um nicht vorhandene Zahlen
\makeatother%

% URLs
\usepackage[hyphens]{url}% URL im Literaturverzeichnis
\urlstyle{same}% Schriftart der URL


%%%%%%%%%%%%%%%%%%%%%%%%%%%%%%%%%%%%%%%%%%%%%%%%%%%%%%%%%%%%%%%%%%%%%%%%%%%%%%%%%%%%%%%%%%%%%%%%%%%%%%%%%%%%%%%%%%%%%%%%%%%%%%%%

%%% Häufig benötigte und nützliche Pakete

% Tabellen
\usepackage{booktabs}% Linien in Tabellen: \toprule (oben), \midrule (innerhalb der Tabelle), \bottomrule (unten)
\usepackage{longtable}% für Tabellen mit Seitenumbruch
\usepackage{ltxtable}% longtable und tabularx (Tabellen über mehrere Seiten, Gesamtbreite einstellbar)
\usepackage{dcolumn}% Tabellenspalte am Dezimaltrenner ausrichten (statt l/c/r): Format D{Dezimaltrenner in Tabelle}{Dezimaltrenner, der ausgegeben werden soll}{Dezimalstellen}

\usepackage{pdflscape}% Querformat ermöglichen
\usepackage{ziffer}% kein Abstand nach Dezimaltrenner

%%%%%%%%%%%%%%%%%%%%%%%%%%%%%%%%%%%%%%%%%%%%%%%%%%%%%%%%%%%%%%%%%%%%%%%%%%%%%%%%%%%%%%%%%%%%%%%%%%%%%%%%%%%%%%%%%%%%%%%%%%%%%%%%

\usepackage[section]{placeins} %prevent floats from being moved over it
\usepackage{flafter} %used to force floats to appear after they are defined
\usepackage{float}

\usepackage{setspace}

% Packages für Code.
\usepackage{listings}
\usepackage{xcolor}

\definecolor{codegreen}{rgb}{0,0.6,0}
\definecolor{codegray}{rgb}{0.5,0.5,0.5}
\definecolor{codepurple}{rgb}{0.58,0,0.82}
\definecolor{backcolour}{rgb}{0.95,0.95,0.92}

\lstdefinestyle{mystyle}{
    backgroundcolor=\color{backcolour},   
    commentstyle=\color{codegreen},
    keywordstyle=\color{magenta},
    %numberstyle=\tiny\color{codegray},
    stringstyle=\color{codepurple},
    basicstyle=\ttfamily\footnotesize,
    breakatwhitespace=false,         
    breaklines=true,                 
    captionpos=b,                    
    keepspaces=true,                 
    numbers=left,                    
    numbersep=5pt,                  
    showspaces=false,                
    showstringspaces=false,
    showtabs=false,                  
    tabsize=2
}

\lstset{style=mystyle}


% Theorems.
\newtheorem{example}{Beispiel}
\newtheorem{definition}{Definition}% Header, kann unverändert übernommen werden
\hyphenation{}% hier nicht korrekte Wörter eingeben: Bindestrich an Trennstellen, Wörter durch Komma getrennt (keine Umlaute)

\begin{document}%
%%%%%%%%%%%%%%%%%%%%%%%%%%%%%%%%%%%%%%%%%%%%%%%%%%%%%%%%%%%%%%%%%%%%%%%%%%%%%%%%%%%%%%%%%%%%%%%%%%%%%%%%%%%%%%%%%%%%%%%%%%%%%%%%
%%%%%%%%%%%%%%%%%%%%%%%%%%%%%%%%%%%%%%%%%%%%%%%%%%%%%%%%%%%%%%%%%%%%%%%%%%%%%%%%%%%%%%%%%%%%%%%%%%%%%%%%%%%%%%%%%%%%%%%%%%%%%%%%

%%% Titelseite
%%%%%%%%%%%%%%%%%%%%%%%%%%%%%%%%%%%%%%%%%%%%%%%%%%%%%%%%%%%%%%%%%%%%%%%%%%%%%%%%%%%%%%%%%%%%%%%%%%%%%%%%%%%%%%%%%%%%%%%%%%%%%%%%
%%%%%%%%%%%%%%%%%%%%%%%%%%%%%%%%%%%%%%%% Vorlage für Statistik-Hausarbeiten: Titelseite %%%%%%%%%%%%%%%%%%%%%%%%%%%%%%%%%%%%%%%%
%%%%%%%%%%%%%%%%%%%%%%%%%%%%%%%%%%%%%%%%%%%%%%%%%%%%%%%%%%%%%%%%%%%%%%%%%%%%%%%%%%%%%%%%%%%%%%%%%%%%%%%%%%%%%%%%%%%%%%%%%%%%%%%%


\thispagestyle{empty}%

{\raggedright%
Otto-Friedrich-Universität Bamberg\\% ggf. Namen der Universität eintragen
Lehrstuhl für Statistik und Ökonometrie\\%
Standort: Bamberg\\% eigenen Standort auswählen
Sommersemester 2022\\% Winter-/Sommersemester eintragen
Vorlesung: Blockseminar Survey-Methodik\\% Titel der Veranstaltung eintragen
Prüfer: Dr. Sara Bleniger\\% Name eintragen
}%

\vspace*{\fill}%
\begin{center}%
    \textbf{\Huge{Statistische Geheimhaltung:}}%
\end{center}%
\begin{center}%
    {\linespread{2.5}
    \textbf{\Huge{Cell Key Methode}}%
    }
\end{center}%
\vfill%

{\vspace*{\stretch{1}}%
\raggedleft%
Joshua Simon\\% eintragen
joshua-guenter.simon@stud.uni-bamberg.de\\% eintragen
Master Survey Statistik, 4. Fachsemester\\% eintragen
Matrikelnummer: 2032411\\%
\today\\}% fügt das aktuelle Datum ein, ggf. manuell ändern


\newpage%

%%%%%%%%%%%%%%%%%%%%%%%%%%%%%%%%%%%%%%%%%%%%%%%%%%%%%%%%%%%%%%%%%%%%%%%%%%%%%%%%%%%%%%%%%%%%%%%%%%%%%%%%%%%%%%%%%%%%%%%%%%%%%%%%% Angaben im Dokument ändern


%%%%%%%%%%%%%%%%%%%%%%%%%%%%%%%%%%%%%%%%%%%%%%%%%%%%%%%%%%%%%%%%%%%%%%%%%%%%%%%%%%%%%%%%%%%%%%%%%%%%%%%%%%%%%%%%%%%%%%%%%%%%%%%%

%%% Verzeichnisse

\tableofcontents% Inhaltsverzeichnis
\newpage%

\listoffigures% Abbildungsverzeichnis
\listoftables% Tabellenverzeichnis
\lstlistoflistings

\newpage%
%%%%%%%%%%%%%%%%%%%%%%%%%%%%%%%%%%%%%%%%%%%%%%%%%%%%%%%%%%%%%%%%%%%%%%%%%%%%%%%%%%%%%%%%%%%%%%%%%%%%%%%%%%%%%%%%%%%%%%%%%%%%%%%%

%%% Darstellender Teil

\onehalfspacing% Zeilenabstand


\section{Einführung}%

Die amtliche Statistik sorgt mit einer Vielzahl an Veröffentlichungen für die Bereitstellung von aufbereiteten statistischen Informationen. Damit geht sie dem Ziel nach, Bürgern, Institutionen und anderen gesellschaftlichen Einrichtungen eine Datengrundlage für die Entscheidungsfindung zu bieten. Weiter dient die amtliche Statistik auch der Politik und der Wissenschaft als Datenquelle. Das Sammeln und Erheben dieser Daten stellt in vielen Fällen einen Eingriff auf das Recht der informationellen Selbstbestimmung für Personen und Entitäten dar. Dieses Recht ist das Fundament des mordernen Datenschutzes und wird über Artikel 2 des Grundgesetztes abgedeckt. Es steht außer Frage, dass dieses Recht besonders schützenswert ist. Demnach steht auch die amtliche Statistik in der Pflicht dieser Verantwortung nachzukommen. Konkret manifestiert sich das Einhalten dieser Plficht in dem sog. Statistikgeheimnis. Aus dem Bundestatistikgesetzt lässt sich hierzu der folgende Absatz aufgreifen (\S 16 Abs. 1 Satz 1 BStatG):

\textit{\glqq Einzelangaben über persönliche und sachliche Verhältnisse, die für eine Bundesstatistik gemacht werden, sind von den Amtsträgern und für den öffentlichen Dinest besonders Verpflichteten, die mit der Durchführung von Bundesstatistiken betraut sind, geheim zu halten, soweit durch besondere Rechtsvorschrift nichts anderes bestimmt ist.\grqq{}}

Konkret möchte man mit der statistischen Geheimhaltung einen Schutz für einzelne Person und Entitäten vor der Offenlegung ihrer sensitven Daten bieten. Dies dient im Weiteren auch der Aufrechterhaltung des Vertrauensverhältnisses zwischen den Befragten und den statistischen Ämtern und erhebenden Einrichtungen. Dies gewährleistet abermals die Zuverlässigkeit der Angaben und der Berichtswilligkeit der Befragten. Die vorausgegangenen Punkte werden in einer Begrüdung zum BStatG erwähnt [\cite{Nickl}]. Ausnahmen von einer Geheimhaltung bestehen nur in Ausnahmefällen, z.B. wenn eine explizite Einwillung zur Veröffentlichung durch den Befragten vorliegt oder wenn sich die Informationen aus allgemein zugänglichen Quellen von öffentlichen Stellen beziehen. Auch die inner-beördliche Übermittlung, Methodenentwicklung, Planungs- und Forschungszwecke werden über das BStatG geregelt.

Im weiteren Verlauf dieser Arbeit sollen Geheimhaltungsverfahren und Geheimhaltungsregeln präsentiert werden, die im Einzelnen die statistische Geheimhaltung gewährleisten und damit dem Statistikgeheimnis der amtlichen Statistik nachkommen. Besondere Beachtung wird dabei der Cell Key Methode (CKM) geschenkt. Dieses Verfahren bietet ein Ansatz, welcher gegenüber anderen Verfahren gut zu Implementieren und zu Automatisieren ist. Gerade dieser Punkt ist in einem immer weiter werdenden technolgischen Umfeld nicht außer Acht zu lassen.

%%%%%%%%%%%%%%%%%%%%%%%%%%%%%%%%%%%%%%%%%%%%%%%%%%%%%%%%%%%%%%%%%%%%%%%%%%%%%%%%%%%%%%%%%%%%%%%%%%%%%%%%%%%%%%%%%%%%%%%%%%%%%%%%%%%%%%%%%%%%

%\newpage
\section{Etablierte Geheimhaltungsverfahren}

In diesem Abschnitt sollen zunächst die grundlegenden Begrifflichkeiten für Geheimhaltungsverfahren und Geheimhaltungsregeln beschrieben werden. 

\subsection{Methodische Grundlagen}

Grundsätzlich gibt es zwei sich unterscheidenede methodische Ansätze, die bei der Geheimhaltung zum Tragen kommen können. Zum einen existieren \textit{informationsreduzierende Methoden}. In der Gattung dieser Verfahren werden durch Aggregation oder Sperrrung kritische Kategorien oder Werte die Aufdeckungsrisiken verhindert. Eine Aggregation meint in diesem Fall das Zusammenfassen zu übergeordneteten Positionen, z.B. durch Summation kleinerer Positionen. Bei einer Sperrung ist auch oft von einer Löschung die Rede. Hier werden gezielt einzelne Werte identifiziert und aus der Tabelle entfernt. Als Kontrast stehen \textit{datenverändernde Methoden} gegenüber. Hier werden durch gezielte Veränderungen der Daten - beispielsweise durch Runden oder Zufallsüberlagerungen - kritische Werte verfälscht, was auch für eine erfolgreiche statistische Geheimhaltung sorgen kann. 

Weiter differenziert man Geheimhaltungsverfahren auch nach dem Zeitpunkt ihrer Durchführung. Die Daten können bereits vor der Tabellierung mit einer Geheimhaltung versehen werden. Man spricht hier von \textit{pre-tabulare Verfahren}. Diese Verfahren werden als Anonymisierung bezeichnet, da die Daten im Vorfeld so verändert werden, dass keine kritischen Ergebnisse resultieren. Oftmals ist diese Art von Geheimhaltung aber nicht ausreichend, weshalb weitere Verfahren im Anschluss angewandt werden müssen. Man spricht nun von \textit{post-tabulare Verfahren}. Ihre Mechanismen werden auf die ferig tabellierten Daten angewandt.


\subsection{Statistische Tabellen}

Gegenstand der Geheimhaltung stellen in dieser Arbeit statistische Tabellen dar. Ein Großteil der Veröffentlichungen der amtlichen Statistik sind selbst - oder beinhalten - statistische Tabellen, welche aus den amtlichen Daten abgeleitet werden. Maßgebend für die Anwendung eines Geheimhaltungsverfahrens ist die Art der zu veröffentlichenden Tabelle, die vorliegt. Man unterscheidet im allgemeinen zwischen \textit{Häufigkeitstabllen} und \textit{Wertetabellen}. Erstere stellen Häufigkeiten oder Fallzahlen dar, z.B. die Anzahl von Frauen und Männern innerhalb einer Universität. Wertetabellen hingegen stellen Wertesummen wie Umsätze dar. Diese unterschiedlichen Kontexte, in denen die Zahlen dieser Tabellen interpretiert werden können, fordern eine natürliche Unterscheidung innerhalb der Geheimhaltung. Es folgen zwei einfache Beispiel für diese Tabellentypen mit rein fiktiven Ausprägungen.

\begin{table}[h]
    \centering
    \begin{tabular}{ r r r r }
        \textbf{Studienfach} \vline & \textbf{männlich} & \textbf{weiblich} & \textbf{insgesamt} \\ 
        \hline
        Bauingenieurwesen \vline & $4$ & $3$ & $7$ \\
        Informatik \vline & $9$ & $12$ & $21$ \\  
        Medizin \vline & $4$ & $1$ & $5$ \\
        Survey Statistik \vline & $10$ & $10$ & $20$ \\
        \hline
        Gesamt \vline & $27$ & $26$ & $53$
    \end{tabular}
    \caption{Beispiel für eine Häufigkeitstabelle}
\end{table}

\begin{table}[h]
    \centering
    \begin{tabular}{ r r r r r }
        \textbf{Brauerei} \vline & \textbf{Mährs Bräu} & \textbf{Schinkerla} & \textbf{Käsmann} & \textbf{Gesamt} \\ 
        \hline
        Umsatz \vline & $600.000$ & $50.000$ & $250.000$ & $900.000$
        \end{tabular}
    \caption{Beispiel für eine Wertetabelle}
\end{table}

%%%%%%%%%%%%%%%%%%%%%%%%%%%%%%%%%%%%%%%%%%%%%%%%%%%%%%%%%%%%%%%%%%%%%%%%%%%%%%%%%%%%%%%%%%%%%%%%%%%%%%%%%%%%%%%%%%%%%%%%%%%%%%%%%%%%%%%%%%%%


\section{Cell Key Methode}

Die im vorherigen Kapitel beschrieben Geheimhaltungsverfahren müssen in der Regel - zumindest bis zu einem gewissen Grad - manuell durchgeführt werden und eine Automatisierung ist eher unfelxibel. Mit der \textit{Cell Key Methode (CKM)} wird ein Geheimhaltungsverfahren präsentiert, welches gut zu automatisieren und vergleichsweise einfach zu implementieren ist. Die Cell Key Methode ist auch als \textit{ABS-Verfahren} bekannt. Der Name stammt von der schöpfenden Institutuion des Verfahrens, dem Australian Bureau of Statistics, ab. Durch die Verwendung von zufallsbasierten Additionen, den sog. Überlagerungen, werden Datenwerte verschleiert. Die CKM zählt damit zu den datenveränderndeden Verfahren. 

\subsection{Methodik}%

Die wichtigsten Bestandteile des Verfahrens werden in [\cite{Enderle}] dargestellt. Ähnlich wie in [\cite{Wipke}] beschrieben, lässt sich nun ein Algorithmus formulieren.

\begin{enumerate}
    \item Erzeugung der Originalwerte mit einem Auswertungstool
    \item Cell-Key-Bestimmung aus Zufallszahlen innerhalb des Auswertungs-Tools
    \item Lookup-Modul
    \begin{enumerate}
        \item Auslesen der Überlagerungswerte aus der Überlagerungsmatrix
        \item Addieren der Überlagerungswerte und Originalwerte
    \end{enumerate}
\end{enumerate}

Die folgende Abbildung visualisiert den schehmatischen Ablauf des zuvor beschrieben Algorithmus.

\begin{figure}[H]
    \begin{center}
        \includegraphics[width=0.85\textwidth]{img/ckm_flow.png}
        \caption{Ablaufdiagramm der Cell Key Methode}
    \end{center}
\end{figure}

\subsubsection{Erzeugung der Originalwerte}

\subsubsection{Cell-Key-Bestimmung}

Jedem Mikrodatensatz wird eine gleichverteile Zufallszahl $r$, dem sog. \textit{Record-Key}, mit $r \sim \mathcal{U}(0, 1)$ zugeordnet. Mit diesen Record-Keys wird dieselbe Auswertungstabelle wie mit den Originalwerten gebildet. Es ergeben sich also Summen von Record-Keys. Von diesen Record-Key Summen werden nur die Nachkommastellen betrachtet. Dieser Wert definiert den \textit{Cell-Key}.

\subsubsection{Lookup-Modul}

\subsection{Besonderheiten der Cell Key Methode}%


%%%%%%%%%%%%%%%%%%%%%%%%%%%%%%%%%%%%%%%%%%%%%%%%%%%%%%%%%%%%%%%%%%%%%%%%%%%%%%%%%%%%%%%%%%%%%%%%%%%%%%%%%%%%%%%%%%%%%%%%%%%%%%%%%%%%%%%%%%%%

%\newpage
\section{Ergebnisse und Auswertungen}

In diesem Kapitel sollen Ergebnisse über die Datenqualität zusammengetragen werden. Die Analyse solcher (Meta-)Daten ist zentral für das Data Engineering, da hiermit das implementierte System validiert und eine mögliche Verbesserung der Datenqualität durch verschiedene Aufbereitungsprozesse gemessen werden kann. Im Anschluss folgt ein kurzer Überblick über die erhobenen Daten selbst.


%%%%%%%%%%%%%%%%%%%%%%%%%%%%%%%%%%%%%%%%%%%%%%%%%%%%%%%%%%%%%%%%%%%%%%%%%%%%%%%%%%%%%%%%%%%%%%%%%%%%%%%%%%%%%%%%%%%%%%%%%%%%%%%%%%%%%%%%%%%%

%\newpage
\section{Zusammenfassung und Fazit}

Damit das Bayerische Landesamt für Statistik seiner gesetzlich vorgeschrieben Pflichten nachkommen kann, sind die unterschiedlichsten Technologien und Verfahren notwendig. Neben dem Erheben der Daten bei den Meldern, gehören auch die Plausibilisierung und die Aufbereitung der Daten zu seinen Aufgaben. Um eine so große Datenmenge effizient zu verarbeiten, wird auf moderne Lösungen aus der Informationstechnologie zurückgriffen. Hierzu zählen verschiedene Skriptsprachen, Datenbanken und Datawarehouses. Um die Qualität der Daten zu wahren, sind neben rein technischer Kontrollen auch sehr fachliche Zusammenhänge zu prüfen und ggf. zu korrigieren. Dies erfordert ein tiefes inhaltliches Verständnis der Daten und ihrer Merkmale. Eine gewisse Interpretation und Analyse sind also auch im Data Engineering notwendig, um die Datenqualität und den Datenfluss zu erhalten, sowie die Daten zu publizieren und damit die Pflicht der amtlichen Statistik zu erfüllen. 


\newpage%
%%%%%%%%%%%%%%%%%%%%%%%%%%%%%%%%%%%%%%%%%%%%%%%%%%%%%%%%%%%%%%%%%%%%%%%%%%%%%%%%%%%%%%%%%%%%%%%%%%%%%%%%%%%%%%%%%%%%%%%%%%%%%%%%
%%%%%%%%%%%%%%%%%%%%%%%%%%%%%%%%%%%%%%%%%%%%%%%%%%%%%%%%%%%%%%%%%%%%%%%%%%%%%%%%%%%%%%%%%%%%%%%%%%%%%%%%%%%%%%%%%%%%%%%%%%%%%%%%

%%% Literaturverzeichnis
%%%%%%%%%%%%%%%%%%%%%%%%%%%%%%%%%%%%%%%%%%%%%%%%%%%%%%%%%%%%%%%%%%%%%%%%%%%%%%%%%%%%%%%%%%%%%%%%%%%%%%%%%%%%%%%%%%%%%%%%%%%%%%%%
%%%%%%%%%%%%%%%%%%%%%%%%%%%%%%%%%% Vorlage für Statistik-Hausarbeiten: Literaturverzeichnis %%%%%%%%%%%%%%%%%%%%%%%%%%%%%%%%%%%%
%%%%%%%%%%%%%%%%%%%%%%%%%%%%%%%%%%%%%%%%%%%%%%%%%%%%%%%%%%%%%%%%%%%%%%%%%%%%%%%%%%%%%%%%%%%%%%%%%%%%%%%%%%%%%%%%%%%%%%%%%%%%%%%%


\begin{thebibliography}{9}
    % beliebige Zahl steht für die Anzahl der Quellen, z.B. 10 Quellen - zweistellige Zahl, 69 Quellen - zweistellige Zahl, 157 Quellen - dreistellige Zahl
\singlespacing%

\bibitem[LfStat]{LfStat}%
Bayerisches Landesamt für Statistik (o.\,J.): \emph{Hochschulen}, online verfügbar unter: \url{https://www.statistik.bayern.de/statistik/bildung_soziales/hochschulen/index.html#link_1} (Zugriff am 29.12.2020).%

\bibitem[Cri20]{cri20}%
Crickard, Paul (2020): \emph{Data Engineering with Python}. Birmingham: Packt Publishing.%

\bibitem[Kim13]{kim13}%
Kimball, Ralph and Ross, Margy (2013): \emph{The Data Warehouse Toolkit: The Definitive Guide to Dimensional Modeling}. Indianapolis: Wiley.%

\bibitem[HStatG]{HStatG}
Statistisches Bundesamt (o.\,J.): \emph{Gesetz über die Statistik für das Hochschulwesen (Hochschulstatistikgesetz – HStatG)}, online verfügbar unter: \url{https://www.destatis.de/DE/Methoden/Rechtsgrundlagen/Statistikbereiche/Inhalte/505_HStatG.pdf?__blob=publicationFile} (Zugriff am 29.12.2020).

\bibitem[Wik19]{T-SQL}%
Wikipedia (2019): \emph{T-SQL - Wikipedia, Die freie Enzyklopädie}, online verfügbar unter: \url{https://de.wikipedia.org/wiki/Transact-SQL} (Zugriff am 10.01.2021).%

\bibitem[Wik21]{Datawarehouse}%
Wikipedia (2021): \emph{Datawarehouse - Wikipedia, Die freie Enzyklopädie}, online verfügbar unter: \url{https://de.wikipedia.org/wiki/Data_Warehouse} (Zugriff am 10.01.2021).%

\bibitem[Wik21]{ETL}%
Wikipedia (2021): \emph{ETL-Prozess - Wikipedia, Die freie Enzyklopädie}, online verfügbar unter: \url{https://de.wikipedia.org/wiki/ETL-Prozess} (Zugriff am 10.01.2021).%


%\bibitem[Name, wie er mit cite im Text zitiert wird]{kuerzel ohne Umlaute, das mit cite angesprochen wird}%
%Nachname, Vorname (Jahr): \emph{Buchtitel}. Ort: Verlag.%

%\bibitem[Autor 2013]{mimimi}%
%Nachname, Vorname (Jahr): Artikel, in: \emph{Zeitschrift} Auflage, S. XX-YY.%

%\bibitem[Autor et\,al. 1999]{magnicht}%
%Nachname, Vorname (o.\,J.): Titel, online verfügbar unter: \url{http...} (Zugriff am \today).%

\end{thebibliography}%

\newpage%

%%%%%%%%%%%%%%%%%%%%%%%%%%%%%%%%%%%%%%%%%%%%%%%%%%%%%%%%%%%%%%%%%%%%%%%%%%%%%%%%%%%%%%%%%%%%%%%%%%%%%%%%%%%%%%%%%%%%%%%%%%%%%%%%%


%%%%%%%%%%%%%%%%%%%%%%%%%%%%%%%%%%%%%%%%%%%%%%%%%%%%%%%%%%%%%%%%%%%%%%%%%%%%%%%%%%%%%%%%%%%%%%%%%%%%%%%%%%%%%%%%%%%%%%%%%%%%%%%%
%%%%%%%%%%%%%%%%%%%%%%%%%%%%%%%%%%%%%%%%%%%%%%%%%%%%%%%%%%%%%%%%%%%%%%%%%%%%%%%%%%%%%%%%%%%%%%%%%%%%%%%%%%%%%%%%%%%%%%%%%%%%%%%%

%%% Wahrheitsgemäße Erklärung

\noindent%
Ich erkläre hiermit, dass ich die Seminararbeit mit dem Titel \emph{Statistische Geheimhaltung: Cell Key Methode} im \emph{Sommersemester 2022} selbständig angefertigt, keine anderen Hilfsmittel als die im Literaturverzeichnis genannten benutzt und alle aus den Quellen und der Literatur wörtlich oder sinngemäß übernommenen Stellen als solche gekennzeichnet habe.%
\bigskip
 
\noindent%
\emph{Bamberg}, den \today\\%
\emph{Unterschrift}%

\end{document}%

%%%%%%%%%%%%%%%%%%%%%%%%%%%%%%%%%%%%%%%%%%%%%%%%%%%%%%%%%%%%%%%%%%%%%%%%%%%%%%%%%%%%%%%%%%%%%%%%%%%%%%%%%%%%%%%%%%%%%%%%%%%%%%%%