%%%%%%%%%%%%%%%%%%%%%%%%%%%%%%%%%%%%%%%%%%%%%%%%%%%%%%%%%%%%%%%%%%%%%%%%%%%%%%%%%%%%%%%%%%%%%%%%%%%%%%%%%%%%%%%%%%%%%%%%%%%%%%%%
%%%%%%%%%%%%%%%%%%%%%%%%%%%%%%%%%% Vorlage für Statistik-Hausarbeiten: Literaturverzeichnis %%%%%%%%%%%%%%%%%%%%%%%%%%%%%%%%%%%%
%%%%%%%%%%%%%%%%%%%%%%%%%%%%%%%%%%%%%%%%%%%%%%%%%%%%%%%%%%%%%%%%%%%%%%%%%%%%%%%%%%%%%%%%%%%%%%%%%%%%%%%%%%%%%%%%%%%%%%%%%%%%%%%%


\begin{thebibliography}{9}
    % beliebige Zahl steht für die Anzahl der Quellen, z.B. 10 Quellen - zweistellige Zahl, 69 Quellen - zweistellige Zahl, 157 Quellen - dreistellige Zahl
\singlespacing%

\bibitem[Enderle, 2019]{Enderle} 
Enderle, Tobias und Meike Vollmar: Geheimhaltung in der Hochschulstatistik. \emph{WISTA} | 6, Statistisches Bundesamt (Destatis), Wiesbaden 2019.

\bibitem[Giessing, 2016]{Giessing} 
Giessing, Sarah: Computational Issues in the Design of Transition Probabilities and Disclosure Risk Estimation for Additive Noise.
\emph{LNCS}, vol. 9867, Springer International Publishing, 2016.

\bibitem[Höhne, 2019]{Höhne} 
Höhne, Jörg und Julia Höninger: Die Cell-Key-Methode – ein Geheimhaltungsverfahren. \emph{Statistische Monatshefte Niedersachsen} 1, 2019.

\bibitem[Nickl, 2019]{Nickl} 
Nickl, Andreas: Datenschutz, Geheimhaltung, Anonymisierung. \emph{Einfuührungsfortbildung} Bayerisches Landesamt für Statistik, Fürth, 2019.

\bibitem[Rothe, 2015-5]{Rothe-1} 
Rothe, Patrick: Statistische Geheimhaltung – Der Schutz vertraulicher Daten in der amtlichen Statistik - Teil 1: Rechtliche und methodische Grundlagen \emph{Bayern in Zahlen} 5, Bayerisches Landesamt für Statistik, München, 2015.

\bibitem[Rothe, 2015-8]{Rothe-2} 
Rothe, Patrick: Statistische Geheimhaltung – Der Schutz vertraulicher Daten in der amtlichen Statistik - Teil 2: Herausforderungen und aktuelle Entwicklungen. \emph{Bayern in Zahlen} 8, Bayerisches Landesamt für Statistik, München, 2015.

\bibitem[Wipke, 2018]{Wipke} 
Wipke, Mirko: Geheimhaltung im Data Warehouse - Prototypische Implementierung von automatisierter Geheimhaltung im Data Warehouse für die amtliche Hochschulstatistik in Bayern. \emph{Bayern in Zahlen} 12, Bayerisches Landesamt für Statistik, Fürth, 2018.


%\bibitem[Name, wie er mit cite im Text zitiert wird]{kuerzel ohne Umlaute, das mit cite angesprochen wird}%
%Nachname, Vorname (Jahr): \emph{Buchtitel}. Ort: Verlag.%

%\bibitem[Autor 2013]{mimimi}%
%Nachname, Vorname (Jahr): Artikel, in: \emph{Zeitschrift} Auflage, S. XX-YY.%

%\bibitem[Autor et\,al. 1999]{magnicht}%
%Nachname, Vorname (o.\,J.): Titel, online verfügbar unter: \url{http...} (Zugriff am \today).%

\end{thebibliography}%

\newpage%

%%%%%%%%%%%%%%%%%%%%%%%%%%%%%%%%%%%%%%%%%%%%%%%%%%%%%%%%%%%%%%%%%%%%%%%%%%%%%%%%%%%%%%%%%%%%%%%%%%%%%%%%%%%%%%%%%%%%%%%%%%%%%%%%