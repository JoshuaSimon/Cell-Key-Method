%%%%%%%%%%%%%%%%%%%%%%%%%%%%%%%%%%%%%%%%%%%%%%%%%%%%%%%%%%%%%%%%%%%%%%%%%%%%%%%%%%%%%%%%%%%%%%%%%%%%%%%%%%%%%%%%%%%%%%%%%%%%%%%%
%%%%%%%%%%%%%%%%%%%%%%%%%%%%%%%%%% Vorlage für Statistik-Hausarbeiten: Literaturverzeichnis %%%%%%%%%%%%%%%%%%%%%%%%%%%%%%%%%%%%
%%%%%%%%%%%%%%%%%%%%%%%%%%%%%%%%%%%%%%%%%%%%%%%%%%%%%%%%%%%%%%%%%%%%%%%%%%%%%%%%%%%%%%%%%%%%%%%%%%%%%%%%%%%%%%%%%%%%%%%%%%%%%%%%


\begin{thebibliography}{9}
    % beliebige Zahl steht für die Anzahl der Quellen, z.B. 10 Quellen - zweistellige Zahl, 69 Quellen - zweistellige Zahl, 157 Quellen - dreistellige Zahl
\singlespacing%

\bibitem[LfStat]{LfStat}%
Bayerisches Landesamt für Statistik (o.\,J.): \emph{Hochschulen}, online verfügbar unter: \url{https://www.statistik.bayern.de/statistik/bildung_soziales/hochschulen/index.html#link_1} (Zugriff am 29.12.2020).%

\bibitem[Cri20]{cri20}%
Crickard, Paul (2020): \emph{Data Engineering with Python}. Birmingham: Packt Publishing.%

\bibitem[Kim13]{kim13}%
Kimball, Ralph and Ross, Margy (2013): \emph{The Data Warehouse Toolkit: The Definitive Guide to Dimensional Modeling}. Indianapolis: Wiley.%

\bibitem[HStatG]{HStatG}
Statistisches Bundesamt (o.\,J.): \emph{Gesetz über die Statistik für das Hochschulwesen (Hochschulstatistikgesetz – HStatG)}, online verfügbar unter: \url{https://www.destatis.de/DE/Methoden/Rechtsgrundlagen/Statistikbereiche/Inhalte/505_HStatG.pdf?__blob=publicationFile} (Zugriff am 29.12.2020).

\bibitem[Wik19]{T-SQL}%
Wikipedia (2019): \emph{T-SQL - Wikipedia, Die freie Enzyklopädie}, online verfügbar unter: \url{https://de.wikipedia.org/wiki/Transact-SQL} (Zugriff am 10.01.2021).%

\bibitem[Wik21]{Datawarehouse}%
Wikipedia (2021): \emph{Datawarehouse - Wikipedia, Die freie Enzyklopädie}, online verfügbar unter: \url{https://de.wikipedia.org/wiki/Data_Warehouse} (Zugriff am 10.01.2021).%

\bibitem[Wik21]{ETL}%
Wikipedia (2021): \emph{ETL-Prozess - Wikipedia, Die freie Enzyklopädie}, online verfügbar unter: \url{https://de.wikipedia.org/wiki/ETL-Prozess} (Zugriff am 10.01.2021).%


%\bibitem[Name, wie er mit cite im Text zitiert wird]{kuerzel ohne Umlaute, das mit cite angesprochen wird}%
%Nachname, Vorname (Jahr): \emph{Buchtitel}. Ort: Verlag.%

%\bibitem[Autor 2013]{mimimi}%
%Nachname, Vorname (Jahr): Artikel, in: \emph{Zeitschrift} Auflage, S. XX-YY.%

%\bibitem[Autor et\,al. 1999]{magnicht}%
%Nachname, Vorname (o.\,J.): Titel, online verfügbar unter: \url{http...} (Zugriff am \today).%

\end{thebibliography}%

\newpage%

%%%%%%%%%%%%%%%%%%%%%%%%%%%%%%%%%%%%%%%%%%%%%%%%%%%%%%%%%%%%%%%%%%%%%%%%%%%%%%%%%%%%%%%%%%%%%%%%%%%%%%%%%%%%%%%%%%%%%%%%%%%%%%%%