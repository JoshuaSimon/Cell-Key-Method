%%%%%%%%%%%%%%%%%%%%%%%%%%%%%%%%%%%%%%%%%%%%%%%%%%%%%%%%%%%%%%%%%%%%%%%%%%%%%%%%%%%%%%%%%%%%%%%%%%%%%%%%%%%%%%%%%%%%%%%%%%%%%%%%
%%%%%%%%%%%%%%%%%%%%%%%%%%%%%%%%%%%%%%%%% Vorlage für Statistik-Hausarbeiten: Header %%%%%%%%%%%%%%%%%%%%%%%%%%%%%%%%%%%%%%%%%%%
%%%%%%%%%%%%%%%%%%%%%%%%%%%%%%%%%%%%%%%%%%%%%%%%%%%%%%%%%%%%%%%%%%%%%%%%%%%%%%%%%%%%%%%%%%%%%%%%%%%%%%%%%%%%%%%%%%%%%%%%%%%%%%%%


%%% Basics

\documentclass[bibtotoc, 12pt, numbers=endperiod, openbib]{scrartcl}%
\usepackage[a4paper, left=30mm, right=30mm, top=20mm, bottom=20mm, includefoot]{geometry}% Seitenränder
\usepackage[utf8]{inputenc}% in TeXnicCenter bitte umändern in: \usepackage[latin1]{inputenc}
\usepackage[T1]{fontenc}%
\usepackage[ngerman]{babel}% in englischsprachigen Arbeiten: \usepackage[english]{babel}
\usepackage{mathptmx}%
\usepackage{amsmath, amsfonts, amssymb}%
\usepackage{graphicx}%


%%%%%%%%%%%%%%%%%%%%%%%%%%%%%%%%%%%%%%%%%%%%%%%%%%%%%%%%%%%%%%%%%%%%%%%%%%%%%%%%%%%%%%%%%%%%%%%%%%%%%%%%%%%%%%%%%%%%%%%%%%%%%%%%

%%% Aussehen allgemein

\usepackage{lmodern}% Schriftart lmodern
\usepackage{setspace}% Zeilenabstand
\pagenumbering{arabic}% arabische Seitenzahlen
\setkomafont{sectioning}{\bfseries}% Serifenschrift für Überschriften
%\usepackage[small]{titlesec}%% verkleinert die Überschriften etwas
%\usepackage[singlelinecheck=0]{caption}% linksbündige Abbildungsüberschriften


%%%%%%%%%%%%%%%%%%%%%%%%%%%%%%%%%%%%%%%%%%%%%%%%%%%%%%%%%%%%%%%%%%%%%%%%%%%%%%%%%%%%%%%%%%%%%%%%%%%%%%%%%%%%%%%%%%%%%%%%%%%%%%%%

%%% Zitate und Literaturverzeichnis

% Zitate mittels \cite[Seitenzahl]{kuerzel}
\usepackage[noadjust]{cite}%
\renewcommand\citeleft{}% Zeichen links vom Zitat, z.B. (, ggf. in die Klammer einfügen
\renewcommand\citeright{}% Zeichen rechts vom Zitat, z.B. ), ggf. in die Klammer einfügen
\renewcommand\citemid{:  }% nach der Seitenzahl folgt ": ", nach Belieben veränderbar

% Literaturverzeichnis
\makeatletter%
\renewcommand{\@biblabel}[1]{}% Literaturverzeichnis nicht nummeriert, keine Klammern um nicht vorhandene Zahlen
\makeatother%

% URLs
\usepackage[hyphens]{url}% URL im Literaturverzeichnis
\urlstyle{same}% Schriftart der URL


%%%%%%%%%%%%%%%%%%%%%%%%%%%%%%%%%%%%%%%%%%%%%%%%%%%%%%%%%%%%%%%%%%%%%%%%%%%%%%%%%%%%%%%%%%%%%%%%%%%%%%%%%%%%%%%%%%%%%%%%%%%%%%%%

%%% Häufig benötigte und nützliche Pakete

% Tabellen
\usepackage{booktabs}% Linien in Tabellen: \toprule (oben), \midrule (innerhalb der Tabelle), \bottomrule (unten)
\usepackage{longtable}% für Tabellen mit Seitenumbruch
\usepackage{ltxtable}% longtable und tabularx (Tabellen über mehrere Seiten, Gesamtbreite einstellbar)
\usepackage{dcolumn}% Tabellenspalte am Dezimaltrenner ausrichten (statt l/c/r): Format D{Dezimaltrenner in Tabelle}{Dezimaltrenner, der ausgegeben werden soll}{Dezimalstellen}

\usepackage{pdflscape}% Querformat ermöglichen
\usepackage{ziffer}% kein Abstand nach Dezimaltrenner

%%%%%%%%%%%%%%%%%%%%%%%%%%%%%%%%%%%%%%%%%%%%%%%%%%%%%%%%%%%%%%%%%%%%%%%%%%%%%%%%%%%%%%%%%%%%%%%%%%%%%%%%%%%%%%%%%%%%%%%%%%%%%%%%

\usepackage[section]{placeins} %prevent floats from being moved over it
\usepackage{flafter} %used to force floats to appear after they are defined
\usepackage{float}

\usepackage{setspace}

% Packages für Code.
\usepackage{listings}
\usepackage{xcolor}

\definecolor{codegreen}{rgb}{0,0.6,0}
\definecolor{codegray}{rgb}{0.5,0.5,0.5}
\definecolor{codepurple}{rgb}{0.58,0,0.82}
\definecolor{backcolour}{rgb}{0.95,0.95,0.92}

\lstdefinestyle{mystyle}{
    backgroundcolor=\color{backcolour},   
    commentstyle=\color{codegreen},
    keywordstyle=\color{magenta},
    %numberstyle=\tiny\color{codegray},
    stringstyle=\color{codepurple},
    basicstyle=\ttfamily\footnotesize,
    breakatwhitespace=false,         
    breaklines=true,                 
    captionpos=b,                    
    keepspaces=true,                 
    numbers=left,                    
    numbersep=5pt,                  
    showspaces=false,                
    showstringspaces=false,
    showtabs=false,                  
    tabsize=2
}

\lstset{style=mystyle}


% Theorems.
\newtheorem{example}{Beispiel}
\newtheorem{definition}{Definition}